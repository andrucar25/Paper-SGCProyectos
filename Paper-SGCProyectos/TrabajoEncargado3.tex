
%
\documentclass[%
 reprint,
 amsmath,amssymb,
 aps,
]{revtex4-1}

\usepackage{graphicx}% Include figure files
\usepackage{dcolumn}% Align table columns on decimal point
\usepackage{bm}% bold math


\begin{document}

\title{Sistema de Gestión para el concurso de Proyectos de la EPIS}
\author{Jose Pastor Mendoza}
\author{Franklin Huichi Contreras}
\author{Andrés De La Barra Vasquez}
\affiliation{%
 Universidad Privada de Tacna \textbackslash Facultad de Ingeniería \textbackslash Escuela Profesional de Ingeniería de Sistemas
}%

\begin{abstract}
\begin{center}
\textbf{Resumen}
\end{center}
Debido a que se usan diferentes herramientas durante la gestión del concurso de proyectos de Escuela Profesional de Ingeniería de Sistemas de la Universidad Privada de Tacna, se desarollará un sistema web denominado "Sistema de Administracion de Proyectos para la EPIS" donde se pueda realizar la gestión y unificar los procesos que conlleva el concurso de proyectos.


\begin{center}
\textbf{Abstract}
\end{center}
Due to the fact that different tools are used during the management of the project contest of the Professional School of Systems Engineering of the Private University of Tacna, a web system called "Project Management System for EPIS" will be developed where the management can be carried out. and unifying the processes involved in the competition for projects.

\end{abstract}



\maketitle

%\tableofcontents

\section {Introducción}

Los sistemas de gestion son un método que se usa para poder administrar, dirigir, operar y observar los cambios de una empresa, institución, o una área determinada con el fin de poder generar resultados positivos y toma de decisiones basadas en datos o hechos concretos. El sistema de gestión a su vez permite tener eficiencia y eficacia en el desempeño de una organización inlcuyendo calidad en la misma gestión del área objetivo.\\
Es por ello que en el presente proyecto va referido al rubro de educación universitaria; ya que se observó que en la Escuela Profesional de Ingeniería de Sistemas realizan algunos procesos de manera no automatizada para la gestión del concurso de proyectos, debido a ello este proyecto se justifica para el beneficio del concurso y la insticuión, la cual implica a los estudiantes, administradores y jurados.

\section{Autores}
\begin{itemize}
\item José Edilberto Pastor Mendoza.
\item Franklin Carlos  Huichi Contreras.
\item Andrés De La Barra Vasquez.
\end{itemize}

\section{Planteamiento del problema}
\subsection{Descripción del problema}
En la Universidad Privada de Tacna, con referencia al concurso de proyectos de la misma institución, no cuenta con una estructura definida en los concursos liberados durante el semestre academico. \\
Los problemas que se encontraron son los siguientes:
\begin{itemize}
\item Los docentes para asegurar la calidad del concurso necesitan revisar que el proyecto no se haya realizado anteriormente, es por ello que se basan en el historial de los proyectos presentados con anterioridad, para ello realizan esta revisión en conjunto con la documentación presentada por los alumnos no sea una copia de documentos de proyectos anteriormente presentados en el concurso para ello realizan una comparación con la documentación de proyectos anteriores, haciendo de esta una tarea tediosa y casi imposible de poder abarcar en todas las documentaciones presentadas.
\item  Por parte del alumno, también el problema seria de que al no saber como se presenta  una documentacion adecuada sufre constantes problemas al momento de entregar, debido a que resulta que no tiene el formato adecuado la documentación presentada al docente. 
\item Otro problema que se observó y que muchos alumnos demandaban era la transparencia en los votos realizados durante el concurso de proyectos, para lo cual ellos querían saber en que criterio de calificación fallaron segun los jurados, en la cual los estudiantes también querían votar por sus proyectos favoritos de manera que estén involucrados como parte de este concurso.
\item Otro problema que se observo es que debido a la gran cantidad de proyectos presentados anteriormente existe ideas parecidas o iguales ya realizadas, esto hace que los alumnos al desconocer la existencia de estos proyectos repiten los mismos haciendo los concursos monótonos y con falta de originalidad.
\end{itemize}

\subsection{Problema}
\subsubsection{General}
¿ Podrá un sistema de gestion resolver los problemas del concurso de proyectos?
\subsubsection{Especificos}
\begin{itemize}
\item ¿Podrá el sistema resolver el problema de los alumnos acerca de la correcta presentación de la documentación del proyecto?
\item ¿El sistema podrá detectar si el documento de un alumno es realizado de manera legal y que no contiene copia alguna de documentación anteriormente presentadas ?
\item ¿El sistema podrá tener transpariencia en su votación en el transcurso del concurso de proyectos ?
\end{itemize}
\subsection{Justificación}
Hoy en día las empresas necesitan tecnología para que sus empleados sean eficientes en su trabajo, de esta manera pueden beneficiarse para obtener una mejor calidad de servicio, mejorar sus procesos y por consiguiente clientes satisfechos.

\subsection{Alcance}
El proyecto se realizará para la gestión de concursos de proyectos en la escuela de la epis, que tendrán los módulos de revisión, sorteo, categorías, cursos, docentes, creación de eventos y el seguimiento de votos para la gestión de el concurso de proyectos. 

%-----------------------------------------------------------------
\section {Objetivos}
\subsection {General}
Resolver los problemas del concurso de proyectos aplicando tecnologías de parte web y móvil. A su vez, integrar todas las actividades involucradas en el proceso del concurso de proyectos ,explicados como problemas, de la EPIS.
\subsection {Especificos}
\begin{itemize}
\item Crear un servicio que nos permita verificar si un documento es único y no una copia del historial de documentos.
\item Crear un servicio que nos permita subir los proyectos a una base de datos.
\item Crear un servicio que nos permita autenticar nuestras credenciales en una base de datos.
\item Integrar estos servicios en una aplicacion móvil y en un sitio web, todo ello haciendo el uso correcto que necesite cada una de las aplicaciones.
\end{itemize}

\section {Desarrollo de la propuesta}
La siguiente propuesta cuenta con tres puntos :
\begin{itemize}
\item Realizar una aplicación web donde los alumnos tendran disponible un repositorio con toda la información sobre los proyectos presentados en ciclos pasados y un formulario donde podran registrarse los proyectos que van a participar del concurso. Además la aplicación web tendra otra vista única para el administrador del sistema, el cual se encargará de revisar los proyectos inscritos que tengan advertencia de plagio, agregará categorías, cursos, asignará docentes al concurso, además de poder visualizar el reporte de votos.  
\item Realizaremos un servicio que detecte si los documentos cumplen con la plantilla establecida parar la inscripción al concurso, ademas de tener un detector de plagios comparando con datos históricos.
\item Relizaremos una aplicación móvil para que los jurados puedan votar.
\end{itemize}
\newpage
Tecnologías que usaremos :
\begin{itemize}
\item Nodejs
\item Reactjs
\item Python
\item MongoDB
\item Firebase
\end{itemize}
\begin{center}
\includegraphics[width=10cm]{./Imagenes/arquitectura2}
\end{center}

%-----------------------------------------------------------------
\section{Conclusiones}

\begin{itemize}
\item La presente propuesta resuelve e integra las actividades totales que conlleva la realización del concurso de proyectos de la EPIS, tales como la inscripción de los participantes, organización de los equipos y también calificación de los jurados. 

\item La implentacion de una funcionalidad extra, que es la posibilidad de los estudiantes espectadores  para votar por el proyecto que más simpatizan,  solicitada por la organizadora del concurso como una posibilidad a implementar a futuro. 

\item Las tecnologías web a aprender y desarrollar durante el proceso del proyecto involucran la actualización e investigación por parte del grupo de trabajo, lo cual cumple uno de los objetivos principales como estudiantes de la carrera de Ingenieria de Sistemas, el cual es ser autodidactas y buscar la constante capacitacion, al tener tecnologias renovadas cada dia.

\item Se espera que el sistema pueda verificar mediante un algoritmo el porcentaje de similitud entre documentos.


 

\end{itemize}

% Bibliografia.
%-----------------------------------------------------------------

%\bibliographystyle{plain}
%\bibliography{Bibliografia}

\end{document}

